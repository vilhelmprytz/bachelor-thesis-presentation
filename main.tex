\documentclass{beamer}
\usepackage[utf8]{inputenc}

\usetheme{Madrid}
\usecolortheme{default}

%------------------------------------------------------------
%This block of code defines the information to appear in the
%Title page
\title[Student Conference] %optional
{Parameter Tuning for Domain Name System Covert Channels}

\subtitle{Evaluating Signature-Based Intrusion Detection System Evasion}

\author[Prytz, Dimitrijevic] % (optional)
{Vilhelm~Prytz \and Filip~Dimitrijevic}

\institute[KTH]{EECS\\ KTH Royal Institute of Technology}

\date[May 28, 2025] % (optional)
{DA150X Student Conference}

\begin{document}

%The next statement creates the title page.
\frame{\titlepage}

%---------------------------------------------------------

\section{Introduction}

%---------------------------------------------------------

\begin{frame}
\frametitle{Introduction}

\begin{itemize}
    \item<1->\textbf{What is DNS?}
    \item<2->Phonebook of the internet: human names → IP addresses. Can be used to associate various data with a domain name.
    \item<3->Essential for internet to work. Often ignored in firewalls.
\end{itemize}

\end{frame}

\begin{frame}
\frametitle{Example Scenario}

\begin{itemize}
    \item<1->\textbf{Imagine a computer system that controls a nuclear reactor}
    \item<2->Assume an attacker has access to this system.
    \item<3->They want to exfiltrate a sensitive file but they cannot, because a firewall is limiting outgoing internet access.
    \item<4->But DNS is not limited! Using a \textbf{DNS covert channel}, they can send the sensitive file over DNS.
    \item<5->How can we detect this covert channel? \textbf{Intrusion Detection System (IDS)}.
\end{itemize}
    
\end{frame}

%---------------------------------------------------------

\begin{frame}
\frametitle{Research Questions}

\begin{itemize}
    \item<1-> Can \textit{Snort} detect \textit{Iodine} covert channels?
    \item<2-> If so, how can the configuration or source code of \textit{Iodine} be altered to avoid detection by \textit{Snort} using established detection rules?
    \item<3-> What generalized conclusions can be drawn about how IDSs detect covert channels?
\end{itemize}
\end{frame}

%---------------------------------------------------------

\section{Method}

%---------------------------------------------------------

\begin{frame}
\frametitle{What are we modifying?}

\begin{itemize}
    \item<1-> \textbf{Record Type}: Iodine employs the NULL record type by default, which is readily detectable through signature-based methods.
    \item<2-> \textbf{EDNS(0) Parameter}: Iodine also makes use of EDNS(0), an extension to DNS.
\end{itemize}
\end{frame}

%---------------------------------------------------------

\begin{frame}
\frametitle{What scenarios are we comparing?}

\begin{itemize}
    \item<1-> \textbf{Baseline}: No parameters modified.
    \item<2-> \textbf{Record Type Parameter}: EDNS(0) Parameter remains unchanged.
    \item<3-> \textbf{EDNS(0) Parameter}: Record Type Parameter remains unchanged.
    \item<4-> \textbf{Both Parameters}: Both parameters are modified.
\end{itemize}
\end{frame}

%---------------------------------------------------------

\begin{frame}
\frametitle{What parameters are we looking at?}

\begin{itemize}
    \item<1-> $\textbf{False Negative Rate (FNR)}= \frac{FN}{FN+TP}$: The proportion of iodine’s covert traffic that Snort fails to detect. A low FNR indicates poor stealth for iodine.
    \item<2-> $\textbf{False Positive Rate (FPR)}= \frac{FP}{FP+FN}$: The proportion of normal (non-tunneled) DNS traffic incorrectly
flagged by Snort as covert.
    \item<3-> \textbf{Bandwidth}: The achieved throughput. That the tunnel is able to send data from the client to the server during the bandwidth test.
\end{itemize}
\end{frame}

%---------------------------------------------------------

\section{Result and Discussion}

\begin{frame}
\frametitle{What is the result?}

\begin{table}[H]
    \centering
    \resizebox{\textwidth}{!}{
    \begin{tabular}{l c c c c c}
    \hline
    \textbf{Metric}     & \textbf{Scenario 1} & \textbf{Scenario 2.1} & \textbf{Scenario 2.2} & \textbf{Scenario 3} & \textbf{Scenario 4} \\ \hline
    False Negative Ratio (FNR)       & 0.82\%                                     & 100\%                                              & 50.36\%                                        & 50.45\%                                    & 100.0\%                                    \\
    False Positive Ratio (FPR)       & 0.0\%                                      & 0.0\%                                              & 0.0\%                                          & 0.0\%                                      & 0.0\%                                      \\
    Bandwidth           & 775 Kbits/sec                              & 879 Kbits/sec                                      & 584 Kbits/sec                                  & 894 Kbits/sec                              & 961 Kbits/sec                              \\ \hline
    \end{tabular}
    }
    \label{tab:result_summary}
\end{table}

\pause

What does this mean?

\pause

\begin{itemize}
    \item<1-> \textit{Iodine} can be modified to evade \textit{Snort}.
    \item<4-> Scenario 2.1 evades \textit{Snort} by simply using another record type.
    \item<5-> The bandwidth remains virtually unchanged across scenarios.
\end{itemize}

\end{frame}

%---------------------------------------------------------

\section{Conclusions}

\begin{frame}
\frametitle{Answering our Research Questions}
  \begin{itemize}
    \item<1-> \textbf{Can \textit{Snort} detect \textit{Iodine} covert channels?}
    \item<2-> Yes, using well-defined rules.

    \item<3-> \textbf{If so, how can the configuration or source code of \textit{Iodine} be altered to avoid detection by \textit{Snort} using established detection rules?}
    \item<4-> By modifying the Record Type parameter and the EDNS(0) parameter.

    \item<5-> \textbf{What generalized conclusions can be drawn about how IDSs detect covert channels?}
    \item<6-> How easy it is to detect \textit{Iodine} depends heavily on the selected ruleset.
  \end{itemize}
\end{frame}

%---------------------------------------------------------

\begin{frame}
\frametitle{Limitations and Problems}

\begin{itemize}
    \item<1-> Selection of parameters.
    \item<2-> Which rules we use.
    \item<3-> Converting Snort2 to Snort3 alerts.
\end{itemize}

\end{frame}

%---------------------------------------------------------

%---------------------------------------------------------
\begin{frame}
\frametitle{Thank you!}

Thank you for listening!

\begin{alertblock}{Acknowledgments}
Jana Tumova, Roberto Guanciale and Fredrik Lindeberg.
\end{alertblock}

\end{frame}
%---------------------------------------------------------

\end{document}